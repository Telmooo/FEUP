	\documentclass[a4paper]{article}
% Hyperreferences
\usepackage{hyperref}
% Minted for Syntax Highlighting
\usepackage{minted}
% Graphics and images
\usepackage{graphicx}
\graphicspath{ {./} }
% Encodings (to render letters with diacritics and special characters)
\usepackage[utf8]{inputenc}
% Language
\usepackage[english]{babel}
% Text Alignment
\usepackage{ragged2e}
% Color
\usepackage{xcolor}
% Math stuff
\usepackage[mathscr]{euscript}
\usepackage{amsmath}
\usepackage{amssymb}
\usepackage{mathtools}
\usepackage{enumitem}
\newcommand{\expnumber}[2]{{#1}\mathrm{e}{#2}} % scientific notation

\usepackage{geometry}
 \geometry{
 a4paper,
 total={170mm,257mm},
 left=20mm,
 top=20mm,
 }

\newtheorem{theorem}{Theorem}

\title{CGRA}
\author{Telmo Baptista}

% Document
\begin{document}
\maketitle
\begin{flushleft}
\newpage
\section{Geometric Transformations}
\subsection{2D Geometric Transformations}
Transformations:
\begin{itemize}
	\item Translation
	\item Scaling
	\item Rotation
\end{itemize}

\subsubsection{Translation}
\begin{equation}
\begin{cases}
	x_T = x + T_x \\
	y_T = y + T_y
\end{cases}
\end{equation}

Matricial form:  
\begin{equation}
\begin{bmatrix}
	x_T \\
	y_T
\end{bmatrix}
=
\begin{bmatrix}
	1 & 0 & T_x \\
	0 & 1 & T_y \\
	0 & 0 & 1
\end{bmatrix}
\begin{bmatrix}
	x \\
	y \\
	1
\end{bmatrix}
\end{equation}

\subsubsection{Scaling}
Regarding the origin:
\begin{equation}
\begin{cases}
	x_S = x \times S_x \\
	y_S = y \times S_y
\end{cases}
\end{equation}

Matricial form:
\begin{equation}
\begin{bmatrix}
	x_S \\
	y_S
\end{bmatrix}
=
\begin{bmatrix}
	S_x & 0 \\
	0 & S_y
\end{bmatrix}
\begin{bmatrix}
	x \\
	y
\end{bmatrix}
\end{equation}

Scaling factor:
\begin{itemize}
	\item $>1$ - increases the object
	\item $<1$ - decreases the object
	\item $S_x=S_y$ - uniform scaling factor $\rightarrow$ doesn't distort object
\end{itemize}

\subsubsection{Rotation}
Around the origin:
\begin{equation}
\begin{cases}
	x = R.cos(\alpha) \\
	y = R.sin(\alpha)
\end{cases}
\end{equation}
\begin{equation}
\begin{cases}
	x_R = R.cos(\alpha+\beta) = R.cos(\alpha).cos(\beta) - R.sin(\alpha).sin(\beta)= x.cos(\beta) - y.sin(\beta) \\
	y_R = R.sin(\alpha+\beta) = R.cos(\alpha).sin(\beta) - R.sin(\alpha).cos(\beta)= x.sin(\beta) + y.cos(\beta)
\end{cases}
\end{equation}

Matricial form:
\begin{equation}
\begin{bmatrix}
	x_R \\
	y_R
\end{bmatrix}
=
\begin{bmatrix}
	cos(\beta) & -sin(\beta) \\
	sin(\beta) & cos(\beta)
\end{bmatrix}
\begin{bmatrix}
	x \\
	y
\end{bmatrix}
\end{equation}

\subsection{3D Geometric Transformations}
Transformations:
\begin{itemize}
	\item Translation
	\item Scaling
	\item Rotation
\end{itemize}

\begin{tabular}{ c c }
 Rotation Axis & Positive Rotation Direction \\
 $x$ & $y \rightarrow z$ \\ 
 $y$ & $z \rightarrow x$ \\  
 $z$ & $x \rightarrow y$
\end{tabular}

\subsubsection{Translation}
\begin{equation}
\begin{cases}
	x_T = x + T_x \\
	y_T = y + T_y \\
	z_T = z + T_z
\end{cases}
\end{equation}

Matricial form:
\begin{equation}
\begin{bmatrix}
	x_T \\
	y_T \\
	z_T \\
	1
\end{bmatrix}
=
\begin{bmatrix}
	1 & 0 & 0 & T_x \\
	0 & 1 & 0 & T_y \\
	0 & 0 & 1 & T_z \\
	0 & 0 & 0 & 1
\end{bmatrix}
\begin{bmatrix}
	x \\
	y \\
	z \\
	1
\end{bmatrix}
\end{equation}

\subsubsection{Scaling}
Regarding the origin:
\begin{equation}
\begin{cases}
	x_S = x \times S_x \\
	y_S = y \times S_y \\
	z_S = z \times S_z
\end{cases}
\end{equation}

Matricial form:
\begin{equation}
\begin{bmatrix}
	x_S \\
	y_S \\
	z_S \\
	1
\end{bmatrix}
=
\begin{bmatrix}
	S_x & 0 & 0 & 0 \\
	0 & S_y & 0 & 0 \\
	0 & 0 & S_z & 0 \\
	0 & 0 & 0 & 1
\end{bmatrix}
\begin{bmatrix}
	x \\
	y \\
	z \\
	1
\end{bmatrix}
\end{equation}

Regarding an arbitrary point:
\begin{equation}
\begin{cases}
	x_S = x \times S_x \\
	y_S = y \times S_y \\
	z_S = z \times S_z
\end{cases}
\end{equation}

Matricial form:
\begin{equation}
T(x_F, y_F, z_F).S(S_x, S_y, S_z).T(-x_F, -y_F, -z_F) =
\begin{bmatrix}
	S_x & 0 & 0 & (1-S_x).x_F \\
	0 & S_y & 0 & (1-S_y).y_F \\
	0 & 0 & S_z & (1-S_z).z_F \\
	0 & 0 & 0 & 1
\end{bmatrix}
\begin{bmatrix}
	x \\
	y \\
	z \\
	1
\end{bmatrix}
\end{equation}

\subsubsection{Rotation}
Around the $z$ axis $\rightarrow$ $z$ constant

\begin{equation}
\begin{cases}
	x_{Rz} = x.cos(\alpha)-y.sin(\alpha) \\
	y_{Rz} = x.sin(\alpha)+y.cos(\alpha) \\
	z_{Rz} = z
\end{cases}
\end{equation}

Matricial form:
\begin{equation}
\begin{bmatrix}
	x_{Rz} \\
	y_{Rz} \\
	z_{Rz} \\
	1
\end{bmatrix}
=
\begin{bmatrix}
	cos(\alpha) & -sin(\alpha) & 0 & 0 \\
	sin(\alpha) & cos(\alpha) & 0 & 0 \\
	0 & 0 & 1 & 0 \\
	0 & 0 & 0 & 1
\end{bmatrix}
\begin{bmatrix}
	x \\
	y \\
	z \\
	1
\end{bmatrix}
\end{equation}

Around the $x$ axis $\rightarrow$ $x$ constant

\begin{equation}
\begin{cases}
	x_{Rz} = x \\
	y_{Rz} = y.cos(\alpha)-z.sin(\alpha) \\
	z_{Rz} = y.sin(\alpha)+z.cos(\alpha)
\end{cases}
\end{equation}

Matricial form:
\begin{equation}
\begin{bmatrix}
	x_{Rz} \\
	y_{Rz} \\
	z_{Rz} \\
	1
\end{bmatrix}
=
\begin{bmatrix}
	1 & 0 & 0 & 0 \\
	0 & cos(\alpha) & -sin(\alpha) & 0 \\
	0 & sin(\alpha) & cos(\alpha) & 0  \\
	0 & 0 & 0 & 1
\end{bmatrix}
\begin{bmatrix}
	x \\
	y \\
	z \\
	1
\end{bmatrix}
\end{equation}

Around the $y$ axis $\rightarrow$ $y$ constant

\begin{equation}
\begin{cases}
	x_{Rz} = x.cos(\alpha)+y.sin(\alpha) \\
	y_{Rz} = y \\
	z_{Rz} = -x.sin(\alpha)+z.cos(\alpha) \\
\end{cases}
\end{equation}

Matricial form:
\begin{equation}
\begin{bmatrix}
	x_{Rz} \\
	y_{Rz} \\
	z_{Rz} \\
	1
\end{bmatrix}
=
\begin{bmatrix}
	cos(\alpha) & 0 & sin(\alpha) & 0 \\
	0 & 1 & 0 & 0 \\
	-sin(\alpha) & 0 & cos(\alpha) & 0 \\
	0 & 0 & 0 & 1
\end{bmatrix}
\begin{bmatrix}
	x \\
	y \\
	z \\
	1
\end{bmatrix}
\end{equation}


\section{Illumination Model}
The lighting models express the lighting components that define the intensity of light reflected by a given surface, allowing the calculation of the color for each surface point of the objects contained in the image

The incident light on the face is reflected in two ways:
\begin{itemize}
	\item Diffuse Reflection - light reflects in all directions, with an equal intensity value, due to the 		roughness of the reflecting surface
	\item Specular Reflection - point sources of light produce over-lit areas on the reflecting surface
\end{itemize}

\subsection{Elementary Illumination Model}

\subsubsection{Ambient Lighting}
Corresponds to the diffuse lighting, whose light from numerous reflections

\begin{equation}
	I = k_a.I_a
\end{equation}
\begin{itemize}
	\item $k_a$ - ambient (diffuse) reflection coefficient of the face, varies between $0$ and $1$
	\item $I$ - observed intensity
\end{itemize}
The intensity $I_a$ is constant in all direction. If we only considered this component to define the light reflected by the object, then all faces would have the same luminous intensity\\
The reflected light is uniform across the face and independent of the observer's position\\
The edges are not distinguishable

\subsubsection{Diffuse Reflection}
The diffuse reflection due to a \textbf{point light source} is calculated according to Lambert's Law: the reflected light intensity depends on the angle of illumination\\
The intensity observed on the object varies, depending on the orientation of the surface and the distance to the light source

\textbf{Note}: The intensity of reflected light doesn't depend on the position of the observer. It depends on the angle of incidence of the light

\begin{equation}
	I = \frac{k_d.I_p}{d+d_0}cos(\theta)
\end{equation}
\begin{equation}
	cos(\theta) = N.L
\end{equation}

Vectors are unitary:
\begin{itemize}
	\item $\theta$ - angle of incidence of the light source
	\item $N$ - normal to the source (unit vector)
	\item $L$ - direction of the illumination beam (incident radius)
	\item $I_p$ - light source intensity
	\item $K_d$ - diffuse reflection coefficient
\end{itemize}

\textbf{Adding the two components}:
\begin{equation}
	I = k_a.I_a + \frac{k_d.I_p}{d+d_0}N.L
\end{equation}

\subsubsection{Specular Reflection/Phong's Model}
Observed reflection on polished surfaces
\begin{itemize}
	\item $R$ - maximum reflection direction
	\item $\alpha$ - angle between $R$ and the direction of the observer $V$
\end{itemize}
\begin{equation}
	I_s = \frac{k_s.I_p}{d+d_0}cos^n(\alpha)
\end{equation}

The specular reflection depends on the position of the observer\\
$k_s$ is a constant that depends on the material, as well as the exponent $n$ (strictly speaking, one should use a function $W(\theta)$ instead of $k_s$)\\
On an ideal surface (ideal mirror), the light reflected only in the $R$ direction\\
On a non-ideal surface, the $R$ direction will have the greatest reflection intensity, the other directions will have lesser intensities\\
The intensity of the specular reflection is proportional to $cos^n(\alpha)$, where $n$ depends on the characteristics of the surface (value 1 for unpolished surfaces and 200 for perfectly polished faces)\\

If $V$ and $R$ are unitary vectors:
\begin{equation}
	I_s = \frac{k_s.I_p}{d+d_0}cos^n(\alpha) = \frac{k_s.I_p}{d+d_0}(V.R)^n
\end{equation}

\subsubsection{Elementary Model}
The expression for the intensity results on:
\begin{equation}
	I = k_a.I_a + I_p [\frac{k_d}{d+d_0}N.L + \frac{k_s}{d+d_0}(R.V)^n]
\end{equation}

Reflection coefficients:
\begin{itemize}
	\item $k_a$ and $k_d$ are commonly the same 
\end{itemize}

Can be broken into components colored (RGB or other):
\begin{itemize}
	\item $I$, $I_a$, $I_p$
	\item $k_a$, $k_d$
	\item $k_s$
	\item $n$
\end{itemize}

\subsubsection{Refraction (for modeling transparent objects)}
When the object is transparent, it is necessary to predict the light that passes through a face, it is called transmitted light or refracted light\\
Because the speed of light is different in different materials, the angle of refraction is different materials, the angle of refraction is different from the angle of incidence

\begin{itemize}
	\item $\eta_i$ - refractive index of air
	\item $\eta_r$ - material refractive index
	\item $\eta$ - is obtained for a given material as the ratio between the speed of light in the void and the speed in the material
\end{itemize}

\textbf{Snell's Law}:
\begin{equation}
	sin(\theta_r) = \frac{\eta_i}{\eta_r}sin(\theta_i)
\end{equation}

\subsubsection{Calculation of Vector R is complex...}
\begin{equation}
	\vec{L} + \vec{R} = \vec{N}.2.\vert\vec{R}\vert.cos(\theta)
\end{equation}
\begin{equation}
	\vec{R} = 2.\vec{N}.(\vec{N}.\vec{L})-\vec{L}
\end{equation}

\section{Shading and Textures}

\subsection{Shading \& Smooth Shading}
\textbf{Objective}: Calculate the color of each point of the visible surface

Solution \textbf{brute-force}: calculate the normal at each point and apply the desired illumination model

\textbf{Different methods}:
\begin{itemize}
	\item Constant Shading
	\item Interpolated Shading = Smooth Shading
	\begin{itemize}
		\item Gouraud Method
		\item Phong Method
	\end{itemize}
\end{itemize}

\subsubsection{Constant Shading}
The color is calculated only for one point of the polygon and replicated in all other points of the same polygon\\
This method is equivalent to the following conditions:
\begin{itemize}
	\item the light source is at infinity, so that $N.L$ is constant at any point of the polygon (parallel rays)
	\item the observer is at infinity so that $R.V$ is constant at any point of the polygon
	\item the face is the flat surface of the model itself and is not an approximation of a curved surface
\end{itemize}

\textbf{The polygonal mesh is noticeable} - Mach Band Effect, with discontinuity of the light function

\subsubsection{Interpolated Shading or Smooth Shading}
In the previous solution, when approaching a curved surface by a polygonal mesh, we found discontinuity in color between adjacent polygons (Mach Band effect, with discontinuity of illumination function)\\
The following solutions will surpass this problem by determining the color of a point based on a interpolation from the vertices of the polygon

\textbf{Requires}: Normals, on the vertices, to the original surface
\begin{enumerate}
	\item Analytical solution - analytical expression of the surface
	\item Approximate solution - Interpolation of the normals of neighbouring polygons
\end{enumerate}

\subsubsection*{Gouraud Method}
\begin{enumerate}
	\item Calculate the color of each vertex using the desired illumination model
	\item Calculate the color of the remaining points of the polygon by bi-linear interpolation
\end{enumerate}

\end{flushleft}
\end{document}